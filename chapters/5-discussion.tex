\chapter{Discussion}
\label{chap:discussion}

\section{Addressing the Research Questions Based on the Experiments}
\info{– Quantitative results vs. baselines}
\info{Ablations on clip length, masking ratio, MAMBA variants}
\info{Error analysis / qualitative examples}


\subsection{RQ1:} 
How does the accuracy of the masked-autoencoder and mamba temporal action spotting model compare to existing \acrlong{sota} methods on the SoccerNet benchmark?
\subsection{RQ2:} 
What are the model’s inference speed and computational requirements compared to T-DEED?
\subsection{RQ3:}
Which types of adaptation can be made to the MAMBA model to increase its precision?

Hyperparameters configured from before, quite well. 
They might be suited to the feature extracted and not mamba directly. Experiment with this


\section{Shortcomings}
didnt fine tune the feature extractor
didnt test on rbk data
no pretraining


\section{Reflections}
mamba should be explored more in the sports action recognition based on litareture review and results. 
a lot of time was spent debugging and setting up the project to run with mamba. 
registered for the competition
asked for too little help. didnt use either github nor discord nor supervisors very much
should have designed a math formula for hyperparameter optimalization
had daily standups with myself, those were useful. especially to see what took a lot of time, what blocked me.
trying to make my own model from scratch was a waste of time
could it be erronous that all clips have the same size because of the last clips in each game not having

\subsection{Sustainability}
VideoMAE is not very good. MAMBA in itself is nice. if you calculate power. i can do something similar with power need as done in visual intelligence, but i do not have all the data in wandb now. 