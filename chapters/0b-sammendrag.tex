\chapter*{Sammendrag}

Denne masteroppgaven undersøker det Mamba-baserte \acrfull{vms} for å detektere handlinger i fotballvideoer. Den sammenligner \acrfull{vms} med \acrfull{tdeed}-modellen på SoccerNet-datasettet. Forskningen evaluerte \acrshort{vms} sin nøyaktighet, hastighet og potensielle forbedringsområder.

Resultatene viser at \acrshort{vms} trener mye raskere (f.eks. 43 minutter i stedet for over 54 timer for \acrshort{tdeed}). Den krever også mindre kraftige \acrfull{gpu}er. Det var imidlertid vanskelig å avgjøre hvilken modell som var mest nøyaktig. Et langt steg for å lage egenskaper for \acrshort{vms} (\acrfull{vmae}, 342 minutter per kamp) betyr at \acrshort{tdeed} for øyeblikket er raskere (18 minutter totalt per kamp) for å behandle nye videoer fra start til slutt. \acrshort{vms} kan gjøre prediksjoner på sekunder hvis man ser bort fra forbehandlingen.

\acrshort{vms}, med sitt S6-design, slo ikke \acrshort{tdeed} for handlingsgjenkjenning i fotball i denne studien. Imidlertid viser dens raske trening og lave databehov at den er lovende. Modellen er spesielt levedyktig når dataressurser er begrenset, eller modeller må oppdateres ofte.

Oppgaven foreslår begrensninger og fremtidig arbeid for å ytterligere forbedre prediksjonsnøyaktigheten. Bærekraftsspørsmål tas opp i diskusjonen.

