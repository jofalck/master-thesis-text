\chapter{Methodology} 
\label{chap:methodology}

\section{Tools and Resources}

\subsection{IDUN}
Video data require large neural networks and computational capacity. \acrfull{idun} is not \acrshort{ntnu}s supercomputer, but a project by the faculties. As of \date{2024-04-24}\footnote{\url{https://www.hpc.ntnu.no/idun/}} it contains 234 \acrfull{gpu}s. \acrshort{gpu}s are very important in accelerated environments such as \acrshort{s6}. 

had to use newer GPUs because of hardware constraints in mamba. what constraints, memory and software/hardware code. but also it sometimes worked. i think for the most cases memory was a problem. 

\subsection{draw.io}
I use this to draw for the PDF, do I talk about that?

\subsection{wandb}
organize experiments and runs. i started using this late on vms. a feature by default on sn teamspotting. i am unsure if this is a useful section. maybe it belongs in background/theory

\subsection{conda}

didnt use very much, just created python environments then installed everything with pip. during the autumn pytorch suggested not using conda anymore, thats the reason i used pip inside conda. 

\section{Data Preprocessing \& Feature Extraction}
\label{sec:preprocessing}

Instead of writing a new model from scratch to fit the input data provided out of the box by soccernet, the data was changed instead. 
thumos14 was picked as goal for data format. 

30\% validation 70\% training

\unsure{simplify the code and make it shorter. only post relevant sections? (remove classes, try except, imports etc}
\info{This code listing is very long. Maybe appendix?}
\lstinputlisting[
    caption={Script to change the label-file to the THUMOS-14 format.},
    label=lst:jsonfile,
    language=Python
]{listings/sn_to_json.py}
the next section is how the raw videos are transformed. 
\todo{also trained with 0.8 split and got better result}

\unsure{the code uses random.random to create a train/val split. to reproduce, should I upload the used data for reproducibility?}

We split each full‐match video into non‐overlapping clips of fixed duration (60 s). Clips are sampled at 25 fps and stored as input sequences $V_k=\{x_t\}_{t=1}^{T_c}$ where $T_c=60\times25$. \todo{edge cases(end of game)}. The duration key is unnecessary in the case of temporal action localization, as shown in the \autoref{lst:uselessduration} which sets a default value if the duration is not explicitly set. 

\lstinputlisting[
    caption={Part of \acrshort{vms} code used to determine duration to be pointless. },
    label=lst:uselessduration,
    language=Python
]{listings/useless_duration.py}

The .json file does not differentiate between full minutes and the edge cases at the end of games. This does not matter because duration parameter doesnt happen. feels like this is repetition. 

Each clip $V_k$ is fed through the pre‐trained VideoMAE-V2 encoder (\autoref{ssec:videomae_v2}) to extract a feature embedding
\[
z_k = \mathrm{VideoMAE\text{-}V2}(V_k)\;\in\;\mathbb{R}^D.
\]
These $D$‐dimensional vectors serve as inputs to all downstream temporal models.

I have to check how I treat the edge cases in the end of the game. The model I use handles inputs of variable length, but doing it this way was a simplificiaton. Also I had no way of knowing a better way to split the videos or any sources on that. 
Future work could be that, or to use overlaps etc. 

VMAE2 is very computationally heavy.

VMAE2 is pretrained on the kinetics dataset, and not customized for football data but rather that type (sports). \todo{write about kinetics dataset in background}

Internvideo which is the base of an experiment where i compare x,1408 vectors from the kinetics model to x,3302 vectors from an bigger masked autoencoder. Same author on VMAE2 and InterVideo. 

The weights from nternvieo 2 is available for download from huggingface, link is findable. 

\section{Model Architectures}
\label{sec:model_architectures}
% … only describe the RNN/TCN/Transformer variants you actually implement 

\subsection{VideoMAE-V2}
\label{ssec:videomae_v2}

VideoMAE‑V2 \cite{wang_videomae_2023} is a self‑supervised masked autoencoder designed specifically to preprocess and compress large‑scale video data before downstream experiments. Building on the original VideoMAE, it introduces a dual‑stage masking schedule: 

\info{feature extractor (hyperparams, masking schedule)}
\begin{itemize}
    \item \emph{Sparse spatiotemporal masking} in early epochs to encourage global context learning over long clips,
    \item \emph{Dense tube masking} in later epochs to refine local motion representations.
\end{itemize}
Each frame is first partitioned into non‑overlapping $P\times P$ patches and tokenized; a binary mask tensor $M\in\{0,1\}^{T\times H/P\times W/P}$ with masking ratio up to 90\% is then generated according to the dual schedule. The visible tokens $x_{\rm vis}=x\odot(1-M)$ are encoded by a standard ViT backbone, while a lightweight three‑layer transformer decoder reconstructs only the masked patches. 

By dynamically varying the masking granularity, VideoMAE‑V2 achieves:
\begin{itemize}
    \item 2× faster convergence compared to VideoMAE,
    \item linear scaling to billion‑parameter models,
    \item robust transferability across action recognition, temporal localization and segmentation tasks.
\end{itemize}

preprocessing wth videomae provides quicker training because features are saved and they are smaller than videos in size

\subsection{mamba}
\label{ssec:mamba}

video mamba suite pretty much sota on papers with code
rdfa26 slitghly better
hardware issues and poor documentation compared to vms prevented it from running

% description of data preprocessing or augmentation (clip length, masking ratios, train/val splits) belongs in Methods

default feature extraction with vmae2

\subsection{\acrfull{tdeed}}

\section{Training Details}
% optimizer, learning rates, augmentation, early‐stopping, etc.

talk about model selection? vms was easier than rdfa-s6 to implement, even though rdfa-s6 perform slightly better. but just slightly. vms had a better backbone, example it included annotations from thumos data. provided better download links. more functional and understandable run file. 

there are config files that show the details. early stopping is implemented. no real agumentation but some in the config.

\section{Temporal model architectures under test (only the ones you implement}

there are some very fancy graphs on wandb i dont understand with gradients

\section{somewhere there should be a section about bayesian hyperparameter search}
i also should really learn this because i've just used it

\section{visualisation}

idk if this deserves a section or subsection but last year the boys had it. i might define more visualisations while I write discussion and results

\section{Evaluation protocols (mAP $\Delta$, runtime measurement)}

mAP 50\% from vms with the mAP from sn. 

runtime I can just calculate from wandb logs, it actually gives the measurement directly. but it depends on gpu.
calculate power used? if thats a metric, or just use the same gpu. 

\section{Environment} 

multiple environments
i used python, but environments also had CUDA code, and c++ code. 
different repositories for different experiments
is vms environment or tool?